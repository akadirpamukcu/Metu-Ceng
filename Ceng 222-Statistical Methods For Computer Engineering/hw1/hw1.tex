\documentclass[12pt]{article}
\usepackage[utf8]{inputenc}
\usepackage{float}
\usepackage{amsmath}


\usepackage[hmargin=3cm,vmargin=6.0cm]{geometry}
\topmargin=-2cm
\addtolength{\textheight}{6.5cm}
\addtolength{\textwidth}{2.0cm}
\setlength{\oddsidemargin}{0.0cm}
\setlength{\evensidemargin}{0.0cm}
\usepackage{amsfonts}

\begin{document}

\section*{Student Information}

Name : Abdulkadir Pamukçu

ID : 2237774


\section*{Answer 1}
\subsection*{a)}
Let call P\{A\} = Probability of picked ball is green and P\{B\} = Probability of select the box X. Probability of picking a green ball given that the selected box is X = P\{A\ $\mid$ B\}\\
\\
P\{A\ $\cap$ B\} = 4/10 $\times$ 1/3\\
P\{B\} = 4/10\\
\[ P\{A\ \mid B\} = \frac{ 4/10 \times 1/3 }{4/10} = \frac{1}{3}   \]\\ 
\\
Hence probability of picking a green ball given that the selected box is X:
$\frac{1}{3}$

\subsection*{b)}
Probability of select a red box from X is selecting X . selecting a red box from x :\\
0.4 $\times$ 2/6 = 4/10 $\times$ 1/3 = 4/30 = 2/15 \\
\\
Probability of select a red box from y is selecting Y . selecting a red box from y :\\
0.6 $\times$ 1/5 = 6/10 $\times$ 1/5 = 6/50 = 3/25\\
\\
Then probability of selecting a red ball is probability of selecting red ball from x + probability of selecting red ball from\\
y: 2/15 + 3/25 = 10/75 + 9/75 = 19/75
\subsection*{c)}
Let call P\{B\} = Probability of picked ball is blue and P\{A\} = Probability of select the box Y. Probability of choosing the box Y given that ball we picked is blue = P\{A\ $\mid$ B\} \\
\\
P\{B\} = Probability of picking blue ball from X + picking a blue ball from Y\\
P\{B\} = 4/10 $\times$ 1/3 + 6/10 $\times$ 2/5 = 28/75\\
\\
P\{A\ $\cap$ B\} =  6/25\\
\\
By conditional probability:\\
P\{A\ $\mid$ B\} =  P\{A\ $\cap$ B\} /  P\{B\}\\
P\{A\ $\mid$ B\} = (6/25) / (28/75)  = 9/14 \\
\\
Hence, probability that we had chosen the box Y given that the ball we picked is blue is 9/14
\section*{Answer 2}

\subsection*{a)}
If $\overline{A}$ and $\overline{B}$ are exhaustive, $\overline{A}  \cup  \overline{B}$ = $\Omega$\\
By definiton 2.8 if $\overline{A}  \cup  \overline{B}$ = $\Omega$ we can say that A  $\cap$  B = $\emptyset$\\
Since if they have some elements in common union of their complements can't be equal to  $\Omega$\\
\\
By definiton if two sets has no elements in common meaning that A  $\cap$  B = $\emptyset$\\ we can say that A and B are mutually exclusive.\\
\\
Hence, A and B are mutually exclusive.
\subsection*{b)}
To disprove the statement given we just need to show one example that contradicts with the statement.\\
\\
Let's take the situation that: A $\supset$ C and A $\cap$ B = $\emptyset$ \\
In this situation $\overline{A}$, $\overline{B}$ and $\overline{C}$, are exhaustive but they are not mutually exclusive because C is a subset of A.\\
\section*{Answer 3}
\subsection*{a)}
There will be to 2 successful dice and 4 unsuccessful dice. And the sample space is equal to 6$^5$.\\
To choose the places of two successful dice out of five,  we got ${5\choose 2}$ and in order to be successful they can take two value: {5,6}. To calculate them we got 2$^2$\\
\\
Also there will be 3 unseccessful dice. And they can take 4 values: {1,2,3,4}. To calculate them we got 4$^3$\\
\\
So that the probability of having exactly two successful dice will be calculated as:\\
\\
\[\frac{{5\choose 2} \times 2^2 \times 4^3}{6^5} = \frac{80}{243} \]\\


\subsection*{b)}
To find probability of having at least two successful dice, we need to sum probability of having exactly one successful dice and probability of having exactly zero successful dice and extract it from 1.\\
\\
Probability of having exactly 1 successful dice:\\
\\
\[\frac{{5\choose 1} \times 2^1 \times 4^4}{6^5} = \frac{80}{243} \]\\
\\
Probability of having exactly 0 successful dice:\\
\\
\[\frac{{5\choose 0} \times 2^0 \times 4^5}{6^5} = \frac{32}{243} \]\\
\\
So that the probability of having exactly two successful dice will be calculated as:\\
\\
\[\ 1 -  (\frac{80}{243} + \frac{32}{243}) = \frac{131}{243} \]
\\
\section*{Answer 4}

\subsection*{a)}
To find P(A=1, C=0) we need to calculate P(1,b,0):\\
P(1,1,0) + P(1,0,0) = 0.06 + 0.09 = 0.15
\subsection*{b)}
To find P(B=1) we need to calculate P(a,1,c): \\
P(1,1,0) + P(1,1,1) + P(0,1,1) + P(0,1,0) = 0.09 + 0.08 + 0.02 + 0.21 = 0.40
\subsection*{c)}
In order to random variables A and B are independent it must be:\\ 
P(A $\cap$ B) = P(A) $\times$ P(B):\\
P(A=1,B=1) = P(A=1) $\times$ P(B=1)\\
P(A=1,B=1) = 0.17 and P(A=1) = 0.55 and P(B=1) = 0.40\\ and P(A=1) $\times$ P(B=1) = 0.22\\
\\
Since P(A=1,B=1) $\neq$ P(A=1) $\times$ P(B=1), A and B are not independent.
\subsection*{d)}
If A  and B are conditionally independent given an event C this equation should be hold: \\
P( A $\cap$ B $\mid$C) = P(A$\mid$C)$\times $P(B$\mid$C)
P( A $\mid$ B, C) = P(A $\mid$ C)\\
P( A $\mid$ B, C) = (P(A $\mid$ C) $\times$ P(B $\mid$ C)) / P(B $\mid$ C) = 0.8\\
P(A$\mid$C) = 0.8 \\
Hence this equation holds we can say that they are independent.


\end{document}

